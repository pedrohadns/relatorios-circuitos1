\documentclass[12pt]{article}
\usepackage[
backend=biber,
style=abnt,
sorting=nyt
]{biblatex}
\usepackage{csquotes}
\addbibresource{bibliografia.bib}
\AtBeginBibliography{\raggedright \footnotesize}
\setlength{\bibitemsep}{\baselineskip}

\usepackage{amsmath} %Para usar símbolos matemáticos

\usepackage[brazil]{babel} %Para usar a linguagem português
\addto\captionsbrazil{
    \renewcommand{\contentsname}{\texorpdfstring{\normalsize SUMÁRIO}{SUMÁRIO}}
    \renewcommand{\listtablename}{\texorpdfstring{\normalsize ÍNDICE DE TABELAS}{ÍNDICE DE TABELAS}}
    \renewcommand{\listfigurename}{\texorpdfstring{\normalsize ÍNDICE DE FIGURAS}{ÍNDICE DE FIGURAS}}
}

\usepackage{subcaption}
\usepackage{float}
\usepackage{graphicx} %Para inserir imagens
\usepackage{geometry} %Para alterar propriedades da página
\geometry{
a4paper,
left=30mm,
top=30mm,
right=20mm,
bottom=20mm
}

\usepackage[hidelinks]{hyperref}

\usepackage{tocloft}
\renewcommand{\cfttoctitlefont}{\hfill}
\renewcommand{\cftaftertoctitle}{\hfill}
\renewcommand{\cftsecfont}{\normalsize\normalfont\bfseries}
\renewcommand{\cftsubsecindent}{0pt}
\renewcommand{\cftsubsecfont}{\normalsize\normalfont}
\renewcommand{\cftsubsubsecindent}{0pt}
\renewcommand{\cftsubsubsecfont}{\normalsize\normalfont\bfseries}
\renewcommand{\cftsecleader}{\cftdotfill{\cftdotsep}}
\renewcommand{\cftlottitlefont}{\hfill} %Centralizar o título da lista de tabelas
\renewcommand{\cftafterlottitle}{\hfill}
\renewcommand{\cfttabpresnum}{Tabela\ } %Adicionar Tabela antes do nome de cada item
\renewcommand{\cfttabaftersnum}{:\ }
\setlength{\cfttabnumwidth}{4.75em}
\renewcommand{\cftdotsep}{1}
\renewcommand{\cftloftitlefont}{\hfill}
\renewcommand{\cftafterloftitle}{\hfill}
\renewcommand{\cftfigpresnum}{Figura\ }
\renewcommand{\cftfigaftersnum}{:\ }
\setlength{\cftfignumwidth}{4.75em}

\usepackage{titlesec}
\titleformat{\section}{\normalsize\bfseries\MakeUppercase}{\thesection}{0.5em}{}
\titleformat{\subsection}{\normalsize\MakeUppercase}{\thesubsection}{0.5em}{}
\titleformat{\subsubsection}{\normalsize\bfseries}{\thesubsubsection}{0.5em}{}

\usepackage{fancyhdr}
\pagestyle{fancy}
\fancyhf{} % Limpa os cabeçalhos e rodapés
\fancyhead[R]{\footnotesize\thepage} % Coloca o número da página no canto superior direito
\fancyfoot[C]{} % Remove qualquer número de página no rodapé
\renewcommand{\headrulewidth}{0pt} % Remove a linha horizontal do cabeçalho
\renewcommand{\footrulewidth}{0pt} % Remove a linha horizontal do rodapé

\usepackage{multicol}
\usepackage{pgfplots}
\pgfplotsset{compat=1.18}
\usepackage[american]{circuitikz}
\usetikzlibrary{external}\tikzexternalize
\usetikzlibrary{positioning}

\usepackage{setspace} %Para alterar o espaçamento das linhas
\setstretch{1.5}

\begin{document}
    \begin{titlepage}
    \begin{center}
        \large
        Universidade Federal do Espírito Santo - UFES\\
        Departamento de Computação e Eletrônica - DCEL\\
        Engenharia de Computação
        
        \vfill
        \textbf{
        Relatório da experiência 01\\
        Resistores\\~\\
        }
        
        Disciplina: Circuitos Elétricos I\\
        Prof. Flávio Duarte Couto Oliveira\\
        
        \vfill
        \begin{flushright}
            Pedro Henrique Alves do Nascimento
        \end{flushright}
        
        \vfill
        Espírito Santo\\
        Dezembro 2024
    \end{center}
    \newpage
\end{titlepage}
    
    \listoftables
    \thispagestyle{empty}
    \newpage

    \listoffigures
    \thispagestyle{empty}
    \newpage

    \tableofcontents
    \thispagestyle{empty}
    \newpage

    \section{Seção}
    \lipsum[1]
    
    \subsection{SUBSEÇÃO}

    \subsubsection{Subsubseção}
    \begin{figure}[h!]
        \centering
        \begin{minipage}{0.5\textwidth}
            \centering
            \caption{Descrição da figura.}
            \includegraphics[width=\textwidth]{example-image}
            \raggedright \footnotesize Fonte: Autor ou fonte da imagem.
            \label{fig:minhaFigura}
        \end{minipage}
    \end{figure}
\newpage

Exemplos circuitos:
\begin{center}\small
    \begin{tikzpicture}[scale=0.8, transform shape]
        \draw (0,0) to [resistor, l=10 $\Omega$] (3,0) to [short, *-] (3,0)
        to [resistor, l_=60 $\Omega$] (3,2)
        to [resistor, l_=10 $\Omega$] (3,4)
        to [resistor, l_=130 $\Omega$] (3,6)
        to [short, *-, f_<=$i_g$] (0,6)
        to [voltage source, v_=600 V, sources/scale=2] (0,0);
        \draw (3,6) to [R, l=34 $\Omega$] (7,6)
        to [short, *-] (7,6)
        to [R, l=72 $\Omega$, f_=$i_0$] (7,3)
        to [R, l=8 $\Omega$] (7,0)
        to [short, -*] (7,0)
        to (7,0) -- (3,0);
        \draw (7,6) to (7,6) -- (10,6)
        to [R, l=20 $\Omega$] (10,0)
        to (10,0) -- (7,0);
    \end{tikzpicture}
\end{center}

\begin{center}
\resizebox{10cm}{!}{
\Large
\begin{tikzpicture}
    \node (A) at (6.5,3) {};
    \node (B) at (8.5,3) {};
    \ctikzset{nodes width=0.08};
    \draw(0,0) to [short, -*] (3,0)
    to [R, l_=3 k$\Omega$, -*] (3,3)
    to [R, l_=2 k$\Omega$, -*] (3,6)
    to [short] (0,6)
    to [current source, l_=4.5 mA, invert, sources/scale=2] (0,0);
    \draw (3,6) to [short] (12,6)
    to [R, l=1.5 k$\Omega$, -*] (12,3)
    to [R, l=6 k$\Omega$] (12,0)
    to [short] (3,0);
    \draw (3,3) to [R, l=1 k$\Omega$, -*] (A) node[label={above: +}] {};
    \draw (12,3) to [R, l_=5 k$\Omega$, -*] (B) node[label={above: -}] {};
    \draw (A) to [open, l=$v_0$] (B);
\end{tikzpicture}}
\end{center} 

Segue a equação \ref{eq: Potência} da Potência:
    \begin{multicols}{2}
        
        \begin{equation}
            P = \frac{dw}{dt}
            \label{eq: Potência}
        \end{equation}
        \columnbreak
        
        \noindent $P$ = Potência, em Watts\\
        $w$ = Trabalho, em Joules\\
        $t$ = Tempo, em segundos\\
    \end{multicols}
    
\begin{table}[H]
    \centering
    \caption{Tabela de exemplo 1}
    \begin{tabular}{|c|c|}
        \hline
        Column 1 & Column 2 \\
        \hline
        Data 1 & Data 2 \\
        \hline
    \end{tabular}
    \label{tab:tabela teste}
\end{table}

\begin{table}[H]
    \centering
    \caption{Tabela de exemplo 2}
    \begin{tabular}{|c|c|}
        \hline
        Column A & Column B \\
        \hline
        Data A & Data B \\
        \hline
    \end{tabular}
\end{table}

\parencite{einstein}
\cite{dirac}

\newpage
\setstretch{1}
\section*{\hfill REFERÊNCIAS BIBLIOGRÁFICAS\hfill}
\addcontentsline{toc}{section}{REFERÊNCIAS BIBLIOGRÁFICAS}
\printbibliography[heading=none]
\end{document}