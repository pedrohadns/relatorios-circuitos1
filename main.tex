\documentclass[12pt]{article}
\usepackage[
backend=biber,
style=abnt,
sorting=nyt
]{biblatex}
\usepackage{csquotes}
\addbibresource{bibliografia.bib}
\AtBeginBibliography{\raggedright \footnotesize}
\setlength{\bibitemsep}{\baselineskip}

\usepackage{amsmath} %Para usar símbolos matemáticos

\usepackage[brazil]{babel} %Para usar a linguagem português
\addto\captionsbrazil{
    \renewcommand{\contentsname}{\texorpdfstring{\normalsize SUMÁRIO}{SUMÁRIO}}
    \renewcommand{\listtablename}{\texorpdfstring{\normalsize ÍNDICE DE TABELAS}{ÍNDICE DE TABELAS}}
    \renewcommand{\listfigurename}{\texorpdfstring{\normalsize ÍNDICE DE FIGURAS}{ÍNDICE DE FIGURAS}}
}

\usepackage{subcaption}
\usepackage{float}
\usepackage{graphicx} %Para inserir imagens
\usepackage{geometry} %Para alterar propriedades da página
\geometry{
a4paper,
left=30mm,
top=30mm,
right=20mm,
bottom=20mm
}

\usepackage[hidelinks]{hyperref}

\usepackage{tocloft}
\renewcommand{\cfttoctitlefont}{\hfill}
\renewcommand{\cftaftertoctitle}{\hfill}
\renewcommand{\cftsecfont}{\normalsize\normalfont\bfseries}
\renewcommand{\cftsubsecindent}{0pt}
\renewcommand{\cftsubsecfont}{\normalsize\normalfont}
\renewcommand{\cftsubsubsecindent}{0pt}
\renewcommand{\cftsubsubsecfont}{\normalsize\normalfont\bfseries}
\renewcommand{\cftsecleader}{\cftdotfill{\cftdotsep}}
\renewcommand{\cftlottitlefont}{\hfill} %Centralizar o título da lista de tabelas
\renewcommand{\cftafterlottitle}{\hfill}
\renewcommand{\cfttabpresnum}{Tabela\ } %Adicionar Tabela antes do nome de cada item
\renewcommand{\cfttabaftersnum}{:\ }
\setlength{\cfttabnumwidth}{4.75em}
\renewcommand{\cftdotsep}{1}
\renewcommand{\cftloftitlefont}{\hfill}
\renewcommand{\cftafterloftitle}{\hfill}
\renewcommand{\cftfigpresnum}{Figura\ }
\renewcommand{\cftfigaftersnum}{:\ }
\setlength{\cftfignumwidth}{4.75em}

\usepackage{titlesec}
\titleformat{\section}{\normalsize\bfseries\MakeUppercase}{\thesection}{0.5em}{}
\titleformat{\subsection}{\normalsize\MakeUppercase}{\thesubsection}{0.5em}{}
\titleformat{\subsubsection}{\normalsize\bfseries}{\thesubsubsection}{0.5em}{}

\usepackage{fancyhdr}
\pagestyle{fancy}
\fancyhf{} % Limpa os cabeçalhos e rodapés
\fancyhead[R]{\footnotesize\thepage} % Coloca o número da página no canto superior direito
\fancyfoot[C]{} % Remove qualquer número de página no rodapé
\renewcommand{\headrulewidth}{0pt} % Remove a linha horizontal do cabeçalho
\renewcommand{\footrulewidth}{0pt} % Remove a linha horizontal do rodapé

\usepackage{multicol}
\usepackage{pgfplots}
\pgfplotsset{compat=1.18}
\usepackage[american]{circuitikz}
\usetikzlibrary{external}\tikzexternalize
\usetikzlibrary{positioning}

\usepackage{setspace} %Para alterar o espaçamento das linhas
\setstretch{1.5}

\begin{document}
    \begin{titlepage}
    \begin{center}
        \large
        Universidade Federal do Espírito Santo - UFES\\
        Departamento de Computação e Eletrônica - DCEL\\
        Engenharia de Computação
        
        \vfill
        \textbf{
        Relatório da experiência 01\\
        Resistores\\~\\
        }
        
        Disciplina: Circuitos Elétricos I\\
        Prof. Flávio Duarte Couto Oliveira\\
        
        \vfill
        \begin{flushright}
            Pedro Henrique Alves do Nascimento
        \end{flushright}
        
        \vfill
        Espírito Santo\\
        Dezembro 2024
    \end{center}
    \newpage
\end{titlepage}
    
    \listoftables
    \thispagestyle{empty}
    \newpage

    \listoffigures
    \thispagestyle{empty}
    \newpage

    \tableofcontents
    \thispagestyle{empty}
    \newpage

    \section*{Pedro Henrique Alves do Nascimento}
    \section{INTRODUÇÃO TEÓRICA}
    \subsection{RESISTORES EM SÉRIE}
    É definido que quando somente dois elementos de circuito estão ligados ao mesmo nó, eles estão em série, é fácil ver, aplicando a Lei de Kirchhoff das Correntes em cada nó, que elementos em série conduzem a mesma corrente \parencite[][, p. 61]{nilsson}, como mostrado na Figura \ref{fig:r-serie}.

    \begin{figure}[H]
        \centering
        \caption{Resistores ligados em série.}
        \begin{tikzpicture}
            \ctikzset{nodes width=0.06}
            \draw (0,0) node[below] {$h$} to [R, l=$R_7$, *-*, v<=$ $] (3,0) node[below] {$g$} to
            [R, l=$R_6$, -*, v<=$ $] (6,0) node[below] {$f$} to
            [R, l=$R_5$, -*, v<=$ $] (9,0) node[below] {$e$} to
            [R, l_=$R_4$, -*, v<=$ $] (9,3) node[above] {$d$} to
            [R, l_=$R_3$, -*, v^<=$ $] (6,3) node[above] {$c$} to
            [R, l_=$R_2$, -*, v^<=$ $] (3,3) node[above] {$b$} to
            [R, l_=$R_1$, v^<=$ $, -*] (0,3) node[above] {$a$} to
            [voltage source, v_=$v_s$, f<=$i_s$, sources/scale=1.5] (0,0);
        \end{tikzpicture}
        \label{fig:r-serie}
    \end{figure}
    
    Além disso, com a Lei de Kirchhoff das Tensões, é possível ver que $n$ resistores em série podem ser substituídos por um único resistor equivalente de resistência $R_{eq}$, conforme mostrados nas Equações \ref{eq:parte-1-serie} e \ref{eq:parte-2-serie} e na Figura \ref{fig:r-serie-equivalente}.
    \begin{gather}
        -v_s+i_sR_1+i_sR_2+i_sR_3+i_sR_4+i_sR_5+i_sR_6+i_sR_7=0 \label{eq:parte-1-serie}\\
        v_s=i_s\left(R_1+R_2+R_3+R_4+R_5+R_6+R_7\right)\implies v_s=i_s\cdot R_{eq} \label{eq:parte-2-serie}
    \end{gather}

    \begin{figure}[H]
        \centering
        \caption{Resistor em série equivalente.}
        \begin{tikzpicture}
            \ctikzset{nodes width=0.06}
            \draw (0,0) to [short, *-] (3,0) to
            [R, l_=$R_{eq}$, v^<=$ $] (3,3) to
            [short, -*] (0,3) node[above] {$a$} to
            [voltage source, v_=$v_s$, f<=$i_s$, sources/scale=1.5] (0,0) node[below] {$h$};
        \end{tikzpicture}
        \label{fig:r-serie-equivalente}
    \end{figure}

    \section{CONCLUSÃO}

    \newpage
    \setstretch{1}
    \section*{\hfill REFERÊNCIAS BIBLIOGRÁFICAS\hfill}
    \addcontentsline{toc}{section}{REFERÊNCIAS BIBLIOGRÁFICAS}
    \printbibliography[heading=none]
\end{document}
